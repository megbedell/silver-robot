\documentclass[12pt]{article}

% math
\newcommand{\given}{\,|\,}
\newcommand{\trans}[1]{{#1}^{\!\mathsf{T}}}

\begin{document}

\section{Introduction}

Types of supernovae are few in number.

Types of nucleosynthetic process are few in number.

Open clusters show great homogeneity, so intra-cloud mixing must be good.

Therefore, stars cannot span a huge space in chemical abundances.

Indeed, this conclusion has been arrived at previously in theory, and
the dimensionality has been measured before in data. The measurements to
date suffer from significant issues: precision, methodology.

Here we test this with the best possible data set at the present day:
All the reasons these twins are good.

\section{Data}

Bedell!

Show (some of) the 900 plots here; they justify the precision claims.

\section{Experiments and results}

PCA first; it is highly suggestive. Take then the opportunity to cricize PCA.

Something like HMF next. State the model and assumptions. Make it such that
there is an excess variance parameter that is being estimated. Etc.
Something like:
\begin{eqnarray}
y_n &=& \mu + W\cdot x_n + \delta_n + \epsilon_n
\\
p(\delta_n\given q^2) &=& N(\delta\given 0, q^2\,I)
\\
p(\epsilon_n) &=& N(\epsilon\given 0, C_n)
\\
p(x_n) &=& N(x_n\given 0, I)
\\
p(y_n\given \mu,W,q^2) &=& N(y_n\given \mu, W\cdot\trans{W}+q^2\,I+C_n)
\quad ,
\end{eqnarray}
where the data point $y_n$ is a $D$-vector,
$W$ is a $D\times K$ rectangular matrix,
$x_n$ is a $K$-vector,
$\delta_n$ is a deviation due to intrinsic variance,
$\epsilon_n$ is a deviation due to measurement noise,
$q^2$ is an intrinsic variance,
$I$ is the $D\times D$ identity tensor,
$C_n$ is the noise variance tensor for data point $y_n$,
and we have marginalized out all but $\mu,W,q^2$ at the end.

Is there any sense in which the eigen-directions look like SNe yields?
Or nucleosynthetic yields from processes?

\section{Discussion}

\end{document}
